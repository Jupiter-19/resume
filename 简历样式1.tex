% !TeX encoding = GBK
\documentclass[a4paper, 11pt]{ctexart}

\usepackage{ctex}
\usepackage{multicol}
\usepackage{color,xcolor}
\usepackage{graphicx}
\usepackage{amsmath}
\usepackage{setspace}
\usepackage{parskip}
\usepackage{enumitem}
\usepackage{fontawesome} % icon库
\usepackage[colorlinks=true,urlcolor = black]{hyperref}

% 页边距设置
\usepackage{geometry}
\geometry{left=1cm, right=1cm, top=1cm, bottom=1cm}

% secion 设置
\CTEXsetup[format={\kaishu \Large\bfseries\color{myThemecolor}}]{section}

% 颜色设置
\usepackage{colortbl}
  \definecolor{mygray}{gray}{0.9}
  \definecolor{mygray2}{gray}{0.6}
 
 
% 自定义 item
\definecolor{myThemecolor}{rgb}{0,0.3,0.5}
\newcommand{\CVitemA}[3]
{
	\textbf{{\kaishu#1}\hfill {\color{gray}#2} }  \\
	\ifx #3
	\else
	\small\fangsong #3\vspace{\parsep}
	\\[-2.0ex]
	\fi 
}

% 脚注
\newcommand\blfootnote[1]{%
	\begingroup
	\renewcommand\thefootnote{}\footnote{#1}%
	\addtocounter{footnote}{-1}%
	\endgroup
}

% 自定义星
\newcommand{\starfouradd}{\faStar\faStar\faStar\faStar\faStarHalfEmpty}
\newcommand{\starfour}{\faStar\faStar\faStar\faStar\faStarO}
\newcommand{\starthreeadd}{\faStar\faStar\faStar\faStarHalfEmpty\faStarO}
\newcommand{\starthree}{\faStar\faStar\faStar\faStarO\faStarO}
\newcommand{\startwoadd}{\faStar\faStar\faStarHalfEmpty\faStarO\faStarO}




 
\begin{document}
\iffalse
  About me 
\fi 
\begin{multicols}{5}
	\hspace{0.0cm}
	\includegraphics[width=3cm]{photo.jpg} 
	
	\begin{center}
	\begin{tabular}{ >{\columncolor{mygray}} c 
					   >{\columncolor{mygray}} l 
			           >{\columncolor{mygray}} c 
			           >{\columncolor{mygray}} l}
	 
	 \rowcolor{mygray}\multicolumn{4}{l}{\Huge \lishu 张少杰}\\
	 
	 %\faVenus 
	 性别:& 男 & 
	 \faPhone & (+86) 13777556325  \\
	 
	 
	 %\faBirthdayCake 
	 生日:& 1996.08.30   \qquad \qquad  \qquad & 
	 \faEnvelopeO & \href{mailto:zhangshaojie@zju.edu.cn}{zhangshaojie@zju.edu.cn}  \\
	 
	 %\faHome  
	 住址:& 浙江省杭州市 & 
	 \faGithub & \url{https://github.com/Jupiter-19}  \qquad \qquad  \qquad  \qquad  \\
	 
	 %\faInstitution 
	 学校:& 浙江大学 & 
	 %\faGlobe 
	 主页:& \url{https://jupiter-19.github.io/}\\
	 
	 %\faGraduationCap 
	 专业:& 计算数学 $\cdot$ 硕士 &
	 兴趣:& 科幻、写博客、随机微分方程 \\
	\end{tabular}
	\end{center}
\end{multicols}



\iffalse
  More
\fi 
\begin{multicols}{2}


% ==================================================================== %
\section*{\faGraduationCap~教育背景} \vspace{-0.25cm}\hrule
\CVitemA
{阿里巴巴达摩院(DAMO)} {2019.03}{冬季大师班(代数几何)}
\CVitemA
{浙江大学(ZJU)} {2018.09 - 2021.03}
{硕士:计算数学;Rank:1/22 \\ 
 主修课程:高维统计方法、随机过程、矩阵计算}
\CVitemA
{浙江大学(ZJU)} {2014.09 - 2018.07}
{本科:数学与应用数学;Rank:15/82\\
 主修课程:泛函分析、数理统计、算法语言、离散数学、不确定性量化中的数值分析}


\vspace{-.8cm}
% ==================================================================== %
\section*{\faCogs ~相关技能} \vspace{-0.25cm}\hrule	
\begin{tabular}{llc}
 应用数学 & 数值分析、优化理论、有限元法 & \starfouradd \\ 
          & 深度学习、统计学习 & \starfour\\ 
 编程语言 & Python  & \starfour \\ 
 		  & C、C++、R、Matlab  & \startwoadd \\ 
 数据分析 & Numpy、Pandas、Scipy &  \starfour \\ 
	      & Pytorch、TensorFlow、Keras  & \starthreeadd  \\
 工具     & LaTeX、Git、Shell & \starthree\\
\end{tabular} 


\vspace{.2cm}

% ==================================================================== %
\section*{\faTrophy ~  个人荣誉} \vspace{-0.25cm}\hrule
\CVitemA
{研究生国家奖学金} {2019.10}{}
\CVitemA
{浙江大学官方微信公众号个人专访} {2019.09}{}
\CVitemA
{阿里巴巴全球数学竞赛\textsc{Top}50} {2018.12}{}
\CVitemA
{浙江大学优秀毕业生} {2018.06}{}
\CVitemA
{浙江大学研究与创新奖学金} {2017.10}{}
\CVitemA
{浙江大学优秀学生学业奖学金} {2016, 2017}{}


\vspace{-.5cm}

% ==================================================================== %
\section*{\faUsers ~实习经历} \vspace{-0.25cm}\hrule	
\CVitemA
{阿里巴巴集团}{2020.05 - 2020.10}
{ $\bullet$ 担任学术合作实习生. \\
  $\bullet$ 借助图神经网络等深度学习技术处理多维数据,重构城市和高速道路的车辆行驶轨迹. 采用GTN网络将车辆数据对匹配精度提升至95\%,重定位“UNKNOW”数据精度为60\%.\\
  $\bullet$  研究深度神经网络的持续学习(\texttt{Continual Learning})技术. 在CIFAR-10和CORe50数据集上使用Pytorch框架复现多篇论文的结果.}
\CVitemA
{华为技术有限公司(杭州研究所)}{2019.09 - 2020.01}
{ $\bullet$ 担任通用软件开发工程师.\\
  $\bullet$ 优化底层库函数,提高高性能计算速度并给出误差上界的证明,速度提升20\%,误差控制在0.5ULPs.\\
  $\bullet$ 重排布Fortran源码循环结构,借助OpenMP实现循环的矢量化与并行化.}
\CVitemA
{杭州星晴资产管理有限公司}{2018.05 - 2018.08}
{ $\bullet$ 担任量化金融实习生.\\
  $\bullet$ 复现国泰君安研报中196个 \texttt{alpha} 因子的效果,根据IC、IR指标缩减因子范围,为高频交易模型提供有效数据支撑.\\
  $\bullet$ 了解金融市场的运作规律,初步设计 \texttt{alpha} 多因子选股模型,模拟盘收益显著跑赢大盘,开户股市交易.}


\vspace{-.5cm}
% ==================================================================== %
\section*{\faUniversity~ 校园经历} \vspace{-0.25cm}\hrule
%\CVitemA{浙江大学本课生课程教学工作}{2018 - 2019}
%{分别担任《高等数学》、《科学计算》课程助教,工作耐心负责,受到老师和学生好评.}
\CVitemA{浙江大学大学生科研训练计划}{2016 - 2017}
{项目名称《基于 \texttt{Markov} 链模型的房地产风险评价》,针对杭州市若干板块进行房价预测与风险估计,利用HMM模型有效发掘出涨幅和风险指数之间的关联,项目评定优秀.}
\CVitemA
{数学建模竞赛,3次}{2016 - 2017}
{2017年美国大学生数学建模竞赛H奖;2016年中国大学生数学建模竞赛三等奖;2016年浙江大学数学建模竞赛二等奖}
%\CVitemA
%{浙江大学学生校友联络协会副部长}{2014 - 2016}
%{组织和参与了第四、第五、第六届“缘定浙大”校友集体婚礼活动志愿者的招募与培训;参加2015浙大学子“走访校友行”暑期社会实践活动.}


\end{multicols}

\vspace{-1cm}

% ==================================================================== %
\section*{\faPaperPlane ~ 自我评价} \vspace{-0.25cm}\hrule
{\small \qquad 
喜欢数学、喜欢技术,注重理论与应用并重. 在学校内学习了大量数学课,拥有扎实的理论基础.
乐于编程,但人生苦短,故用 \texttt{Python}. 乐于分享,欣赏“开源”精神,在简书、\texttt{Github Pages} 等平台写博客. 乐于学习新的知识与技能.
}


% ==================================================================== %
\blfootnote{
	\color{mygray2}
	\faCopyright ~ \textsc{Shaojie Zhang}, ~~
	%\faCreativeCommons ~ 
	\faGithub ~ jupiter-19/resume, ~~ by \LaTeX, ~~
	\faCalendar ~ \today.
}

\end{document}
